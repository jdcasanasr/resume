\documentclass[10pt]{article}

\usepackage[letterpaper, total={7.5in, 10in}]{geometry}
\usepackage{multicol}
\usepackage[T1]{fontenc}
\usepackage[sfdefault, light]{roboto}
\usepackage{fontawesome}
\usepackage{hyperref}

% Change bullet style for the first and second levels of nested lists.
\renewcommand{\labelitemi}{\tiny $\bullet$}
\renewcommand{\labelitemii}{\tiny $\circ$}

\title{ \vspace{-2em} \Huge{Juan-Daniel Casañas-Roque} \\
		\huge{Electronics \& Computing Engineer} \\
		\begin{tabular}{ccc}
			\centering
			\small{\faicon{envelope} jdaniel.casanasr@gmail.com} &
			\centering
			\small{\faicon{phone} (+52) 5536531062} &
			\centering
			\small{\faicon{github} jdcasanasr}
		\end{tabular}
		\rule{\linewidth}{1pt}
		\vspace{-3em}}
\author{}
\date{}

\begin{document}
	\pagenumbering{gobble}
	\maketitle
	
	\section*{\Large{Education} \hrulefill}
	\textbf{Master in Science in Computing Engineering}
			\hfill 2020 - 2022 \\
	\textit{Computing Research Center - 
	National Polytechnic Institute
			\hfill Mexico City, Mexico} \\
	
	\noindent
	\textbf{Bachelor in Electronic Engineering}
			\hfill 2013 - 2018 \\
	\textit{Metropolitan Autonomous University Azcapotzalco
			\hfill Mexico City, Mexico} \\

	\section*{\Large{Experience} \hrulefill}
	\textbf{Assistant Professor}
			\hfill 2020 - 2022 \\
	\textit{Computing Research Center - 
			National Polytechnic Institute
			\hfill Mexico City, Mexico}

	\vspace{2em}

	\begin{minipage}{5.5in}
		\begin{itemize}
			\item Lectured on vector architectures to fellow and 
					first-year students and volunteered to explain
					various other topics as part of the subjects
					\emph{Processor Architecture} and
					\emph{Advanced Processor Architecture}.
					
			\item Exposed the contents of several research papers and led its 
					discussion.

			\item Proposed original ideas for a vector extension and
					an exception handler, to be used in the
					\emph{Lagarto} project.
		\end{itemize}
	\end{minipage}

	\vspace{3em}

	\noindent
	\textbf{Master in Science Thesis}
			\hfill 2020 - 2022 \\
	\textit{Computing Research Center - 
			National Polytechnic Institute
			\hfill Mexico City, Mexico}
	
	\vspace{2em}

	\begin{minipage}{5.5in}
		\begin{itemize}
			\item Titled \emph{"Dragonfang: An Embedded, General Purpose, RISC-V
				Based Vector Extension for the Lagarto Hun Processor"}.

			\item Developed a vector execution module for the \emph{Lagarto}
				project
			(\url{https://www.hub-innovacion.cic.ipn.mx/index.php/bienvenidos}),
				based upon the RISC-V Instruction Set Architecture
				(\url{https://riscv.org/}), capable of processing multiple
				elements within a single, 64-bit word.

			\item As part of the latter, designed and implemented the following
				modules, using SystemVerilog:

				\begin{itemize}
					\item A 32x64 bit register file with support for
						register groups of 1, 2, 4 or 8 registers.

					\item A vector execution unit, formed by a cluster of 10
						integer and 10 floating-point units.

					\item A vector masking unit.
				\end{itemize}

			\item Evaluated functionality for integer modules via testbenches
				and Modelsim.
		\end{itemize}
	\end{minipage}

	\newpage

	\noindent
	\textbf{Social Service Group Coordinator}
			\hfill 2021 \\
	\textit{Computing Research Center - 
			National Polytechnic Institute
			\hfill Mexico City, Mexico}
	
	\vspace{2em}

	\begin{minipage}{5.5in}
		\begin{itemize}
			\item Took charge of the \emph{Glucometer} project, where proposals
				for the development of a glucometer where evaluated and
				developed by social-service workers, namely:

				\begin{itemize}
					\item A mobile solution using colormetric strips reactive to
						saliva and a smartphone camera.

					\item A Surface Acoustic Wave (SAW) solution, where a
						development board was used for the processing and
						display of the obtained signals.
				\end{itemize}
			
			\item Exposed results in a research poster as part of the 2021
				edition of the reunion of the National Computing Network in
				Mexico City (remote).
		\end{itemize}
	\end{minipage}

	\vspace{3em}

	\noindent
	\textbf{Bachelor Thesis}
			\hfill 2018 \\
	\textit{Metropolitan Autonomous University Azcapotzalco
			\hfill Mexico City, Mexico}

	\vspace{2em}

	\begin{minipage}{5.5in}
		\begin{itemize}
			\item Titled \emph{"System for the Recovery and Processing of
				Electrocardiographic Signals"}.

			\item Developed an autonomous and mobile,
				recovery-and-display-capable ECG system with a cost-accessible
				approach.

			\item Designed and implemented an analogue interface, composed of
				over-the-counter sensors and electronic filters -band-pass and
				60-Hz-tuned notch-.

			\item Designed, routed and fabricated a home-made PCB using KiCAD.

			\item Coded in Arduino C a digital interface, capable of displaying
				the obtained signals in a small, monochromatic, LCD screen.
		\end{itemize}
	\end{minipage}

	\vspace{3em}

	\noindent
	\textbf{Social Service Worker}
			\hfill 2017 - 2018 \\
	\textit{Metropolitan Autonomous University Azcapotzalco
			\hfill Mexico City, Mexico}

	\vspace{2em}

	\begin{minipage}{5.5in}
		\begin{itemize}
			\item Worked on various tasks in the \emph{Technology Development
				and Incubation Laboratory}, such as:

				\begin{itemize}
					\item Development of a battery-recharge circuit for an
						electric wheelchair.

					\item Design and development of a driver circuit for a
						bipolar stepper motor.

					\item Miscellaneous analogue and digital circuits, including
						design, PCB fabrication and soldering.
				\end{itemize}
		\end{itemize}
	\end{minipage}

	\newpage

\begin{multicols}{2}
	\section*{\Large{Languages} \hrulefill}
	
	\noindent
	\textbf{Spanish} \\
	\textit{Native} \\
	
	\noindent
	\textbf{English} \\
	\textit{Advanced}

	\vspace{2em}

	\noindent
	\begin{minipage}{3in}
		\begin{itemize}
			\item C1 level of the Common European Framework of Reference for
				Languages (CEFR).
			\item Very proficient in all 5 language skills.
			\item Can hold conversations entirely in english and express myself
				in public with remarkable pronunciation.
			\item Can write technical and creative documents comfortably.
		\end{itemize}
	\end{minipage}

	\vspace{2em}

	\section*{\Large{Soft Skills} \hrulefill}
	\begin{itemize}
		\item \textit{Effective Exposition of Scientific Themes}
		\item \textit{Technical \& Creative Writing}
		\item \textit{Public Speaking}
		\item \textit{Self-Taught}
		\item \textit{Problem-Solving}
		\item \textit{Critical Thinking}
		\item \textit{Perseverant}
		\item \textit{Proactive}
		\item \textit{Creative}
	\end{itemize}

	\columnbreak

	\section*{\Large{Hard Skills} \hrulefill}

	\textbf{Version Control} \\
	\textit{Git} \\

	\noindent
	\textbf{Programming} \\
	\textit{C/C++, Python, Java, MATLAB/Octave, Bash} \\

	\noindent
	\textbf{Hardware Modelling} \\
	\textit{Verilog, SystemVerilog} \\

	\noindent
	\textbf{Hardware Platforms} \\
	\textit{FPGA, Arduino} \\

	\noindent
	\textbf{Document Preparation} \\
	\LaTeX, Markdown \\

	\noindent
	\textbf{Operating Systems} \\
	\textit{Windows, GNU/Linux} \\

	\noindent
	\textbf{Text Editor} \\
	\textit{Vim} \\

	\noindent
	\textbf{Development Platforms \& Software} \\
	\textit{Intel Quartus, Verilator, KiCAD} \\

	\noindent
	\textbf{Simulation Platforms} \\
	\textit{ModelSim, LTSpice} \\

\end{multicols}

\end{document}
